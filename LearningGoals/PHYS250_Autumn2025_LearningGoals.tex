%----------------------------------------------------
% Packages to use
%----------------------------------------------------
\documentclass[12pt]{article}
\usepackage{hyperref}
\usepackage[pdftex]{graphicx}
\usepackage{multirow}
\usepackage{xspace}

\hypersetup{
  colorlinks=true,
  linkcolor=red,
  citecolor=red,
  urlcolor=blue
}

%----------------------------------------------------
% Page Setup
%----------------------------------------------------
\usepackage{setspace}
\usepackage[margin=0.90in]{geometry}
\linespread{1.3}
%\doublespacing
%\addtolength{\hoffset}{-2.5cm}
%\addtolength{\voffset}{-2.5cm}
%\addtolength{\textwidth}{2.0cm}
%\addtolength{\textheight}{4.0cm}

%----------------------------------------------------
% New commands
%----------------------------------------------------
\input{../Slides/newcommands.sty}

%----------------------------------------------------
% Header and Footer Information
%----------------------------------------------------
\usepackage{fancyhdr}
\pagestyle{fancy}
\lhead{PHYS 250, Autumn Quarter 2025}
\rhead{\today, v0.1}

%----------------------------------------------------
% Course Number, Course Title, Prof Name, Location
%----------------------------------------------------
\title{\sc Physics 250: Computational Physics\\Learning Goals}
\author{\textbf{Instructor:} David W. Miller, MCP 245, \\ 773-702-7671, \href{mailto:David.W.Miller@uchicago.edu}{David.W.Miller@uchicago.edu}}
\date{}


\begin{document}

\maketitle

\thispagestyle{fancy}

%----------------------------------------------------
% Learning Goals
%----------------------------------------------------

\noindent This document differs from a syllabus in that it is not a list of nouns (topics covered), but a list of verbs that indicate what students are \textit{expected to be able \textbf{to do} after taking this course}.

\subsection*{Introduction}

\textbf{The Nature of Science:} Science is all about models. We look at something in real life and try to make a model of it. We can use this model to predict future (or new) events in real life. If the model doesn't agree with real data, we change the model. Repeat forever. [\href{https://www.wired.com/2015/11/what-computational-physics-is-really-about/}{From Wired}]

%--------------------------------
\subsection*{Learning Goals}
%--------------------------------

  \begin{itemize}
    \item \bluebf{Identify models} that benefit from or require \bluebf{computational/numerical approaches} and tools to either develop and understand or to evaluate. 
    \item Develop an \bluebf{algorithmic approach} to addressing those problems computationally, including understanding when and why certain approaches are relevant and important, or irrelevant and misplaced.
    \item Build familiarity with \bluebf{common computational models and algorithms} and numerical approaches to typical problems.
    \item Use \bluebf{modern, high-level programming languages} to implement the computational algorithms. 
    \item Use modern software tools for \bluebf{developing, preserving, disseminating, and expanding} on those solutions.
    \item \bluebf{Apply computational tools} to contemporary physics questions, and develop the skills needed to \bluebf{participate in a modern research lab} that employs computational approaches.
    \item Use AI/LLM tools \bluebf{intentionally and critically} to support computational problem-solving, while demonstrating \bluebf{independent reasoning} and the ability to verify results.
    \item Evaluate and refine AI/LLM-generated outputs by \bluebf{testing, debugging, and adapting} them to align with both algorithmic logic and physical principles.
    \item \bluebf{Document and reflect on AI/LLM use} in computational workflows, articulating when and why the tool was used, and how it contributed (or failed to contribute) to understanding.
  \end{itemize}

\end{document}