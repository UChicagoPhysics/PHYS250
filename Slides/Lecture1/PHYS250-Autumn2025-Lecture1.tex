%----------------------------------------------------
% Setup Beamer
%----------------------------------------------------
\documentclass[hyperref={colorlinks=true}]{beamer}

%----------------------------------------------------
% Packages to use
%----------------------------------------------------
\input{../packages.sty}

%----------------------------------------------------
% Setup Theme
%----------------------------------------------------
\input{../theme.sty}

%----------------------------------------------------
% Table of Contents at each section transition
%----------------------------------------------------

\AtBeginSection[]
{
   \begin{frame}
       \frametitle{Outline}
       \setcounter{tocdepth}{1}
       \tableofcontents[currentsection]
   \end{frame}
}

%----------------------------------------------------
% Colors
%----------------------------------------------------
\input{../mycolors.sty}

%----------------------------------------------------
% Style, formatting, and new commands
%----------------------------------------------------
\newcommand{\CourseYear}   {2025}
\newcommand{\CanvasURL}    {https://canvas.uchicago.edu/courses/66016}
\newcommand{\CanvasLink}   {\href{\CanvasURL}{\CanvasURL}}
\newcommand{\GitHubURL}    {https://github.com/UChicagoPhysics/PHYS250}
\newcommand{\GitHubLink}   {\href{\GitHubURL}{\GitHubURL}}
\newcommand{\PlatformURL}  {https://binderhub.pile.uchicago.edu/}
\newcommand{\PlatformLink} {\href{\PlatformURL}{\PlatformURL}}
\newcommand{\PiazzaURL}    {https://canvas.uchicago.edu/courses/66016/discussion\_topics}
\newcommand{\PiazzaLink}   {\href{\PiazzaURL}{\PiazzaURL}}
\input{../newcommands.sty}

%----------------------------------------------------
% Set paths for plots and images
%----------------------------------------------------
\input{../paths.sty}


%-----------------------------------------------------------------------------------------
% Title: [Column]{Title}
%-----------------------------------------------------------------------------------------
\title[PHYS 250 (Autumn 2025) -- Lecture 1]{Introduction to Computational Physics}

%-----------------------------------------------------------------------------------------
% SubTitle: [Column]{Subtitle}
%-----------------------------------------------------------------------------------------
\subtitle{PHYS 250 (Autumn 2025) -- Lecture 1}

%-----------------------------------------------------------------------------------------
% Author: [SubAuthor]{Author}
%-----------------------------------------------------------------------------------------
\author[D.W.~Miller]{David Miller}

%----------------------------------------------------
% Institute: [SubInst]{Institute}
%----------------------------------------------------
\institute[EFI, Chicago] 
{
  Department of Physics, Enrico Fermi Institute\\
  Kavli Institute for Cosmological Physics, University of Chicago
}

%----------------------------------------------------
% Institute: [SubInst]{Institute}
%----------------------------------------------------
\date[September 30, 2025]{September 30, 2025}

\subject{PHYS 250 Lecture}

\begin{document}

%==========================================================================================
% TITLE PAGE
%==========================================================================================

{
\begin{frame}
  \titlepage
\end{frame}
}

%==========================================================================================
\section[Introduction to PHYS 250]{Introduction}
%==========================================================================================

%-----------------------------------------------------------------------------------------
\subsection[Course Description]{Course Description}
%-----------------------------------------------------------------------------------------

\begin{frame}%[shrink=1]
  \frametitle{Computational Physics (PHYS 250)}
    
  \begin{ucblock}{Course Description PHYS 250 (\href{http://collegecatalog.uchicago.edu/search/?P=PHYS\%2025000}{link to Course Catalog})}
    This course introduces the use of computers in the physical sciences. After an introduction to programming basics, we cover numerical solutions to fundamental types of problems, including cellular automatons, artificial neural networks, computer simulations of complex systems, and finite element analysis. Additional topics may include an introduction to graphical programming, with applications to data acquisition and device control.
  \end{ucblock}
  
  \vspace{0.5cm}
  
  \centering
  
  \bluebf{There are an infinite number of paths that we might follow and still not deviate from this description. I therefore would like to lay out some of the principles that will guide me, and us, in how we navigate through those many possibilities.}
  
\end{frame}

%-----------------------------------------------------------------------------------------
\subsection[Philosophy and Learning Goals]{Philosophy and Learning Goals}
%-----------------------------------------------------------------------------------------

{\setbeamercolor{background canvas}{bg=black}
\begin{frame}%[shrink=1]
  %\frametitle{Patterns}

  \begin{figure}
    \includegraphics<1>[width=\textwidth]{\sciPath/Physics/CMB/Planck_CMB_black_background_fullwidth.pdf}
    \includegraphics<2>[width=0.7\textwidth]{\sciPath/Physics/CMB/CMBPowerSpectrum-PlanckWMAPACTSPT.pdf}
  \end{figure}

  \textcolor{white}{\textbf{The Nature of Science:} Science is all about models. We look at something in real life and try to make a model of it. We can use this model to predict future (or new) events in real life. If the model doesn't agree with real data, we change the model. Repeat forever. [\href{https://www.wired.com/2015/11/what-computational-physics-is-really-about/}{From Wired}]}	  
	  
\end{frame}
}

%---------------------------------------------------------------------------------------------
\begin{frame}%[shrink=1]
  \frametitle{Theory and Practice (or...Theory and Experiment)}

  We often divide the world of science (and perhaps even moreso, Physics) into two deeply related and intertwined but often distinct categories of activities and research:
  
  \begin{itemize}
    \item \bluebf{Theoretical:} building models
    \item \bluebf{Experimental:} testing models
  \end{itemize}	  
	  
  \centering \textbf{We use computers to do \textit{both} of these crucial activities!} 
  
  \flushleft In doing so, we often blur the distinction between the two, as well.
	  
\end{frame}

%-----------------------------------------------------------------------------------------

\begin{frame}%[shrink=1]
  \frametitle{Overarching Learning Goals for this Quarter}

  \begin{itemize}
    \item \bluebf{Identify models} that benefit from or require \bluebf{computational/numerical approaches} and tools to either develop and understand or to evalaute. 
    \item Develop an \bluebf{algorithmic approach} to addressing those problems computationally.
    \item Familiarity with \bluebf{common computational models and algorithms} and numerical approaches to typical problems.
    \item Use \bluebf{modern, high-level programming languages} to implement the computational algorithms. 
    \item Use modern software tools for \bluebf{developing, preserving, disseminating, and expanding} on those solutions.
    \item \bluebf{Apply computational tools} to contemporary physics questions.
  \end{itemize}
  
\end{frame}

%-----------------------------------------------------------------------------------------

\begin{frame}%[shrink=1]
  \frametitle{The role of the physicist}

  \textit{``The computer is incredibly fast, accurate, and stupid. Man is incredibly slow, inaccurate, and brilliant. The marriage of the two is a force beyond calculation.''}
  
  \vspace{0.4cm}
  
  $\rightarrow$ Maybe this is a quote from Albert Einstein, maybe it is from \href{https://en.wikipedia.org/wiki/Leo\_Cherne}{Leo Cherne}...I'm not sure, but it's certainly true.
  
  \centering
  \includegraphics[width=0.6\textwidth]{MachineBrain.jpg}
  
\end{frame}

%-----------------------------------------------------------------------------------------
\subsection[Course Syllabus]{Course Syllabus}
%-----------------------------------------------------------------------------------------

\begin{frame}%[shrink=1]
  \frametitle{Some of the specific problems that we will tackle}

  There are literally hundreds of texts, and even more webpages and software repos, for computational physics, methods, numerical recipes. 
  
  $\rightarrow$ \bluebf{I have chosen a subset of ubiquitous problems and which are the ``bread and butter'' of nearly all domains in physics and beyond}.

  \begin{columns}
   
    \column{0.5\textwidth}
    
      \begin{itemize}
        \item Programming basics and Software design concepts
        \item Basic visualization
        \item Random numbers, errors
        \item Ising model, Metropolis algo
        \item Minimization 
      \end{itemize}
    
    \column{0.5\textwidth}
  
      \begin{itemize}
        \item Monte Carlo method
        \item Ordinary differential equations
        \item Partial differential equations
        \item Fourier transforms
        \item Data analysis techniques
        \item Neural networks
      \end{itemize}
  
  \end{columns}
  
  \vspace{0.5cm}
  
  \centering \textbf{I hope you walk out of this class able to effectively engage with most of the computational physics issues and tools in a modern research group.}
  
\end{frame}



%==========================================================================================
\section[Computational approaches generally]{Computational approaches generally}
%==========================================================================================

%-----------------------------------------------------------------------------------------
\subsection[Algorithmic and process-based thinking]{Algorithmic and process-based thinking}
%-----------------------------------------------------------------------------------------

\begin{frame}%[shrink=1]
  \frametitle{Algorithmic and process-based thinking}
    
  You probably already learned in your first year courses that certain approaches to problem solving in physics are important to efficiently identifying solutions and paths through a particular problem. 
  
  \vspace{1cm}
  
  Before we get into any code, let's discuss the \bluebf{thought process behind developing a computational model or approach to problem solving}. 
    
\end{frame}

%-----------------------------------------------------------------------------------------

\begin{frame}[fragile]
  \frametitle{Thinking like a machine}
  \framesubtitle{From the Kinder \& Nelson text (see list in a few slides)}
    
  \setbeamercovered{transparent}  
    
  \pause
  
  \begin{ucpythonblock}{Human-readable (high level) instructions to start a car}
    Put the key in the ignition
    Turn the key until the engine starts, 
        then let go
    Push the button on the shifter and move it 
        to REVERSE
    ...
  \end{ucpythonblock}

  \pause

  \begin{ucpythonblock}{Bug!!!}
    Press down the left pedal # Needed for some cars
  \end{ucpythonblock}
  
  \pause

  \begin{ucpythonblock}{Machine-readable (low level) instructions}
    Grab the wide end of the key
    Insert pointed end of key into slot on lower 
        right side of steering column
    Rotate key clockwise about long axis when viewed 
        toward the pointed end
  \end{ucpythonblock}
  
  
\end{frame}

%-----------------------------------------------------------------------------------------

\begin{frame}[fragile]
  \frametitle{Logical vs. algorithmic instructions}
    
  \setbeamercovered{transparent}  
    
  Let's say you want to write an algorithm to calculate the area of a circle. 
  
  \begin{enumerate}[<+->]
    \item Write ``pseudocode'' or the \alertbf{logical instructions} for the algorithm
    \item Extend pseudocode with \alertbf{parameters and input/output}
    \item Specify algorithm \alertbf{explicitly}
  \end{enumerate}
    
  \begin{ucpythonblock}{Pseudocode algorithm for area calculation}
calculate area of circle  # Do this computer!
  \end{ucpythonblock}

  \pause

  \begin{ucpythonblock}{Better algorithm for area calculation}
read radius              # Input
calculate area of circle # Numeric
print area               # Output
  \end{ucpythonblock}
  
  \pause

  \begin{ucpythonblock}{Actual algorithm for area calculation}
radius = input("Specify radius")  # Input
pi     = 3.141593                 # Set constant
area   = pi  * radius^2           # Algorithm
print("Area = " + area)           # Output 
  \end{ucpythonblock}
  
  
\end{frame}

%-----------------------------------------------------------------------------------------
\subsection[Computational algorithm development cycles]{Computational algorithm development cycles}
%-----------------------------------------------------------------------------------------

\begin{frame}[shrink=10]
  \frametitle{Translate algorithmic thinking into a development process}
  %\framesubtitle{Concepts adopted from colleague Salvatore Rappoccio at SUNY Buffalo}
  
  We can adapt to the need for algorithmic thinking by adopting a process for developing algorithms, computational approaches, and  software generally.
  
  \begin{center}
  
  %\column{0.8\textwidth}
  
  \begin{shadowblock}{0.7\textwidth}
  
  \setlength{\leftmargini}{40pt}
  
  \begin{enumerate}
    \item[Step 1:] write the algorithm down on paper
    \item[Step 2:] think
    \begin{itemize}
      \item[If:] you don't understand everything
      \item[Then:] \texttt{goto} Step 1
      \item[Else:] \texttt{continue}
    \end{itemize}
    \item[Step 3:] write pseudocode
    \item[Step 4:] think
    \begin{itemize}
      \item[If:] you don't understand everything
      \item[Then:] \texttt{goto} Step 3
      \item[Else:] \texttt{continue}
    \end{itemize}
    \item[Step 5:] write actual code
    \item[Step 6:] test code with unit tests
    \begin{itemize}
      \item[If:] code does not unit checks
      \item[Then:] \texttt{goto} Step 5
      \item[Else:] \texttt{continue}
    \end{itemize}
    \item[Step 7:] publish!
  \end{enumerate}
  \end{shadowblock}

  \end{center}

\end{frame}

%-----------------------------------------------------------------------------------------

\begin{frame}%[shrink=10]
  \frametitle{Alas, sometimes, it's a bit more complicated than that...}
  
  \includegraphics[width=0.95\textwidth]{\physPath/ParticlePhysicsWorkflow.png}

\end{frame}

%==========================================================================================
\section[Software]{Software}
%==========================================================================================

%-----------------------------------------------------------------------------------------
\subsection[Version Control Software Systems]{Version Control Software Systems}
%-----------------------------------------------------------------------------------------

\begin{frame}%[shrink=10]
  \frametitle{Version control}
  
  \begin{columns}
  
    \column{0.5\textwidth}
    
      \begin{itemize}
        \item The most important message of this slide is simple...\alertbf{Use a software version control system for all of your code}
        \begin{itemize}
          \item And that means now...not tomorrow or next week
          \item Because if you wait until you need it, it will be too late
        \end{itemize}
      \end{itemize}
      
      \pause
      
      \centering
      
      \includegraphics[width=0.75\columnwidth]{/Users/fizisist/Pictures/Random/Git-Logo-2Color.png}
      \includegraphics[width=0.75\columnwidth]{/Users/fizisist/Pictures/Random/GitHub_Logo.png}
    
    \column{0.5\textwidth}
    
      \includegraphics[width=0.95\textwidth]{/Users/fizisist/Pictures/Random/version-control.jpg}
  
  \end{columns}

\end{frame}

%-----------------------------------------------------------------------------------------

\begin{frame}%[shrink=10]
  \frametitle{A brief history of version control}
  
  \begin{itemize}
    \item \bluebf{The first version control systems were designed to be used on large systems where everyone logged into the same machine}
    \begin{itemize}
      \item They tracked code on the same filesystem where it lived (e.g., in a subdirectory)
      \item SCCS and RCS are examples
    \end{itemize}
    \item \bluebf{Then client-server systems were developed, so that developers could work on their own machines}
    \begin{itemize}
      \item Checking code into a central server to share and collaborate
      \item CVS and SVN are examples
    \end{itemize}
    \item  \bluebf{More recently distributed version control systems have arisen}
    \begin{itemize}
      \item These are decentralised, so everyone has a complete copy of the repository
      \item Gives a lot of freedom to developers to share and merge as they like, so liked very much by the open source community
      \item git, mercurial and bit keeper are examples
    \end{itemize}
  \end{itemize}
  
\end{frame}

%-----------------------------------------------------------------------------------------

\begin{frame}%[shrink=10]
  \frametitle{\git, \github, \& \gitlab }
  \framesubtitle{\url{https://git-scm.com}, \url{https://github.com}, \url{https://about.gitlab.com}}
  
  \begin{itemize}
    \item \bluebf{\git is the most popular open source version control system}
    \begin{itemize}
      \item can host huge projects (Linux Kernel, LHC experiment software, etc)
      \item scales very well and it's extremely fast and powerful
      \item very flexible (= complex)
    \end{itemize}
    \item \bluebf{Distributed version control systems (\git) are great, but they're made even better by using a social coding site (\github or \gitlab)}
    \item \bluebf{These sites allow developers:}
    \begin{itemize}
      \item browse code easily
      \item compare different versions
      \item take copies (a.k.a. fork)
      \item offer patches back to upstream repositories
      \begin{itemize}
        \item And discuss and review these patches before acceptance
      \end{itemize}
      \item even build websites
    \end{itemize}
    \item \bluebf{The best known social coding site is GitHub, but there are others, e.g. BitBucket and GitLab}
    \begin{itemize}
      \item Familiarity with \git/\github/\gitlab will serve you well, trust me
    \end{itemize}
  \end{itemize}
  
\end{frame}

%-----------------------------------------------------------------------------------------

\begin{frame}%[shrink=10]
  \frametitle{ATLAS Experiment analysis software package on \github}
  
  \includegraphics[width=0.95\textwidth]{xAODAnaHelpers.png}

\end{frame}

%-----------------------------------------------------------------------------------------

\begin{frame}%[shrink=10]
  \frametitle{\github \& \gitlab resources }
  
  \github is a free resource as long as your code remains public, although as a University Student, you can free additional resources (next page). The Physical Sciences Division (PSD) at UChicago hosts a \bluebf{private} \gitlab repository.

  \begin{itemize}
    \item \url{https://psdcomputing.uchicago.edu/page/psd-repo}
  \end{itemize}
  
  \includegraphics[width=0.95\textwidth]{PSDRepo.png}

\end{frame}

%-----------------------------------------------------------------------------------------

\begin{frame}%[shrink=10]
  \frametitle{\github Student Developer Pack}
  \framesubtitle{\url{https://education.github.com/pack\#offers}}
  
  \begin{columns}
  
    \column{0.3\textwidth}
    
      You get free unlimited private repositories as a student! 
    
    \column{0.5\textwidth}
    
      \includegraphics[width=0.95\textwidth]{GitHub-StudentDeveloperPack.png}
  
  \end{columns}

\end{frame}

%-----------------------------------------------------------------------------------------

\begin{frame}%[shrink=10]
  \frametitle{PHYS 250 \github}
  \framesubtitle{\url{https://github.com/UChicagoPhysics/PHYS250}}
  
  Course materials are hosted in the \github \texttt{UChicagoPhysics} repository
  
  \begin{center}
    \includegraphics[width=0.85\textwidth]{PHYS250-GitHub.png}
  \end{center}

  \vspace{-0.5cm}

  \begin{itemize}
    \item Slides (e.g. \textit{these!}), syllabi, learning goals, and code examples
    \item Stable versions will be cross-posted to \href{\CanvasURL}{Canvas} as well.
    \item Homework submission will be done via \github (\textit{instructions to come})
  \end{itemize}
  
  
\end{frame}

%-----------------------------------------------------------------------------------------

\begin{frame}%[shrink=10]
  \frametitle{\github Classroom}
  \framesubtitle{\url{https://classroom.github.com/}}
  
  Course materials are hosted in the \github \texttt{UChicagoPhysics} repository, but the assignments will be distributed via \github Classroom
  
  \begin{itemize}
    \item I create a repository with the assignment
    \item With the click of a button, that repository gets distributed to all of you as your own private repository
    \item The deadline is set, and the lass commit at that point is ``submitted'' to me as your homework 
  \end{itemize}
  
  
\end{frame}

%-----------------------------------------------------------------------------------------

\begin{frame}%[shrink=10]
  \frametitle{\github Classroom}
  \framesubtitle{\url{https://classroom.github.com/}}
  
    \begin{columns}
  
    \column{0.5\textwidth}
    
      \includegraphics[width=0.95\columnwidth]{GitHubClassroom-DistributingAssignments.pdf}
    
    \pause
    
    \column{0.5\textwidth}
    
      \includegraphics[width=0.95\columnwidth]{GitHubClassroom-GradingAssignments.pdf}
  
  \end{columns}
  
  \centering {\tiny Source: \href{https://arxiv.org/abs/1811.02021}{arXiv:1811.02021} }
  
  \vspace{0.2cm}
  
  \centering \alertbf{You need a \github account, so please sign up ASAP!}
  
\end{frame}


%-----------------------------------------------------------------------------------------

\begin{frame}%[shrink=10]
  \frametitle{Operating systems and platforms}
  
  \begin{itemize}
    \item Operating systems (like Windows or Mac OS)
    \item We will generally be using UNIX/LINUX (an offshoot of UNIX)
    \begin{itemize}
      \item Invented by Linus Torvalds in 1991
      \item Mac OS X is built upon LINUX
    \end{itemize}
    \item I have arranged for a ``minicourse'' from CSIL on Friday (info in later slides)
    \begin{itemize}
      \item Also have an EdX course (free) from the creator here:
      \item \url{https://www.edx.org/course/introduction-linux-linuxfoundationx-lfs101x-0\#!}
    \end{itemize}
    \item Linux is completely open source
    \begin{itemize}
      \item you can modify it at your will and debuggers are also free
    \end{itemize}
    \item Predominantly command line tools
  \end{itemize}
  
  
\end{frame}

%-----------------------------------------------------------------------------------------

\begin{frame}%[shrink=10]
  \frametitle{Linux ``shell''}
  
  \begin{itemize}
    \item We will be using an interface to Linux called a ``shell''
    \item It is a command-line interpreter : you type, it executes
    \item Two major options are \texttt{bash} (as in, smash) and \texttt{csh} (like ``sea shell'', modern version is ``tcsh'', ``tea sea shell'')
    \begin{itemize}
      \item Only real difference: environment variables syntax
      \item \texttt{bash:} \texttt{export X=value}
      \item \texttt{csh:} \texttt{setenv X value}
    \end{itemize}
  \end{itemize}
  
\end{frame}

%-----------------------------------------------------------------------------------------

\begin{frame}[fragile]
  \frametitle{Shell basics}
  
  \begin{ucbashblock}{Listing directory contents : \texttt{ls}, like ``list''}
> ls
Examples/ LearningGoals/ README.md Slides/ 
Syllabus/ global.sty
  \end{ucbashblock}
  
  \begin{ucbashblock}{Copy: \texttt{cp}}
> cp stuff.txt stuff1.txt
  \end{ucbashblock}
  
  \begin{ucbashblock}{Where am I?: \texttt{pwd}, \texttt{cd}}
> pwd
/ComputationalPhysics/PHYS250/PHYS250-Fall2018
> ls
Examples/ LearningGoals/ README.md Slides/ Syllabus/
> cd Examples/
> ls
HelloGaussian.ipynb
HelloGaussian.py
Introduction_to_Jupyter_Notebooks_and_Python.ipynb
  \end{ucbashblock}  
\end{frame}


%-----------------------------------------------------------------------------------------
\subsection[Why python?]{Why python?}
%-----------------------------------------------------------------------------------------

\begin{frame}%[shrink=1]
  \frametitle{Python as our programming language}
    
  \begin{ucblock}{From \url{www.python.org}}
    Python is a dynamic object-oriented programming language that can be used for many kinds of software development. It offers strong support for integration with other languages and tools, comes with extensive standard libraries, and can be learned in a few days.
  \end{ucblock}
  
  \bluebf{And in my own words:}
  
  \begin{itemize}
    \item It is \bluebf{ubiquitous, flexible, useful, and \textit{relatively} easy}
  \end{itemize}
    
  From \href{https://spectrum.ieee.org/static/interactive-the-top-programming-languages-2018}{IEEE Spectrum rankings (July 2018)}:

  \begin{center}
    \includegraphics[width=0.45\textwidth]{Python-Trending.png}
    \includegraphics[width=0.45\textwidth]{Python-Jobs.png}
  \end{center}  
    
\end{frame}

%-----------------------------------------------------------------------------------------
\subsection[Python Introduction]{Python Introduction}
%-----------------------------------------------------------------------------------------

\begin{frame}[fragile]
  \frametitle{Hello world!}

  \begin{ucpythonblock}{Interactive in the python interpreter}
python
>>> print "hello world"
hello
>>> CTRL-d  # to exit python
  \end{ucpythonblock}
  
  \begin{ucpythonblock}{From a script (containing the above print line):}
python helloworld.py
  \end{ucpythonblock}
  
  \begin{ucpythonblock}{Self-running script:}
#!/usr/bin/env python
# This script prints hello to the screen
print "hello world"
  \end{ucpythonblock}
  
  \begin{ucpythonblock}{}
chmod +x helloworld.py
./helloworld.py
hello world
  \end{ucpythonblock}
  
\end{frame}

%-----------------------------------------------------------------------------------------

\begin{frame}[fragile]
  \frametitle{Python syntax}

  \begin{itemize}
    \item Everything after a hash (\texttt{\#}) is a comment (until the end of the line)
    \item A statement ends at the end of the line
    \item Multiple statements on the same line are separated by a semicolon (\texttt{;})
    \item A block of code is defined by its equal indentation
    \begin{itemize}
       \item Don't use tabs, always use spaces! (tab = 8 spaces)
    \end{itemize}
    \item A backslash at the end of a line joins it with the next line - like in shell scripts
    \item An unclosed \texttt{()}, \texttt{[]} or \texttt{\{\}} pair also continues to the next line(s) until the closing \texttt{)}, \texttt{]} or \texttt{\}}
  \end{itemize}
  
\end{frame}

%-----------------------------------------------------------------------------------------

\begin{frame}[fragile]
  \frametitle{Numerical types}

  \begin{itemize}
    \item \texttt{int}: integer (at least 32bit)
    \item \texttt{float}: floating point, like C++ double
    \item complex
    \item \texttt{str}: string (constant), like \texttt{C++} \texttt{const char*}
    \item \texttt{bool}: True or False
    \item \texttt{list}: a vector, like \texttt{C++} \texttt{std::vector}
    \item \texttt{tuple}: constant list
    \item \texttt{dict}: a map, like \texttt{C++} \texttt{std::map}
  \end{itemize}
  
  \begin{ucpythonblock}{You can ask an object for it's type with the '\texttt{type}' function:}
>>> type("hello")
<type 'str'>
>>> type("hello").__name__
'str'  
  \end{ucpythonblock}
  
\end{frame}


%-----------------------------------------------------------------------------------------

\begin{frame}[fragile]
  \frametitle{Operators}

  \begin{itemize}
    \item \texttt{+, -, *, /, **}: addition, subtraction, multiplication, division, power 
    \item \texttt{+=, -=, *=, /=}: operator and assignment in one go (as C++)
    \item \texttt{\%}: modulus (int), format (str)
    \item \texttt{or, and}: logical \texttt{OR} and \texttt{AND}
    \item \texttt{not}: logical negation
    \item \texttt{<, <=, >, >=}: comparison
    \item \texttt{==, !=}: equality, not equality
    \item \texttt{is, is not}: object identity (pointer comparison) 
    \item \texttt{in, not in}: membership test
    \item \texttt{|, \^{}, \&, $\sim$, <<, >>}: bitwise operators, as in C++ 
    \item \texttt{X < Y < Z}: True if \texttt{Y} is in between \texttt{X} and \texttt{Z}
  \end{itemize}

\end{frame}

%-----------------------------------------------------------------------------------------

\begin{frame}[fragile]
  \frametitle{Lists (I)}

  In my opinion, python's great advantage is \bluebf{list comprehension}.

    \begin{ucpythonblock}{List basics}
v = []   # empty list
v = list()  # empty list
v = [ 1, 2, 4, 5 ]  ; v = [ 'a', 'b', 'c' ]
v = range(4,10,2) # results in [ 4,6,8 ]
v = [ 4, 2.5, 'Hi', [ 1,3,5 ] ]  # can mix types    
    \end{ucpythonblock}

    \begin{ucpythonblock}{Append elements}
>>> v.append( 70 )
>>> print v  
    \end{ucpythonblock}
    
    \begin{ucpythonblock}{Concatenation}
>>> v += [ 'some', 'more', 'elements' ]
>>> v  # shows the object
    \end{ucpythonblock}
    
    \begin{ucpythonblock}{Removal of elements}
>>> v.remove(2.5)
>>> del v[0]
    \end{ucpythonblock}

\end{frame}


%-----------------------------------------------------------------------------------------

\begin{frame}[fragile]
  \frametitle{Lists (II)}

    \begin{ucpythonblock}{Element acces read/write}
>>> v[0]
'hi'
>>> v[0] = 'hey'
>>> v[-1] # last element. Negative = count from the end
>>> v[1:3] # subrange by index (start index, one-beyond-last index) 
    \end{ucpythonblock}

    \begin{ucpythonblock}{Test if an element is in a list (or not)}
>>> if 4 in v:
...   print "Found it"
Found it
>>> if 200 not in v:
...   print "Not found"
Not found
    \end{ucpythonblock}

\end{frame}

%-----------------------------------------------------------------------------------------

\begin{frame}[fragile]
  \frametitle{\texttt{for} and \texttt{while} loops}

  The \texttt{for} statement iterates through a collection, iterable object or generator function.
  
  The \texttt{while} statement merely loops until a condition is \texttt{False}.

    \begin{ucpythonblock}{Iterate over list}
fruits = ["apple", "banana", "cherry"]
for x in fruits:
  print(x) 
    \end{ucpythonblock}

    \begin{ucpythonblock}{Iterate using built-in \texttt{range} function}
for x in range(0, 3):
  print "We're on time %d" % (x)
    \end{ucpythonblock}


    \begin{ucpythonblock}{Iterate until a condition is met}
count = 0
while count < 5:
  print(count)
  count += 1  # Same as: count = count + 1    
    \end{ucpythonblock}

\end{frame}

%-----------------------------------------------------------------------------------------

\begin{frame}[fragile]
  \frametitle{Putting lists and loops together is amazing (and complex)}

    \begin{ucpythonblock}{Filter one list into another (the ``old'' way)}
newlist = []
for i in oldlist:
    if filter(i):
        newlist.append(function(i))
    \end{ucpythonblock}

    \begin{ucpythonblock}{List comprehension (the ``pythonic'' way)}
newlist = [function(i) for i in oldlist if filter(i)]
    \end{ucpythonblock}

where \texttt{filter} and \texttt{function} just perform ``some'' operation on the list elements. Basically, the syntax is:
    \begin{ucpythonblock}{}
[ expression for item in list if conditional ] 
    \end{ucpythonblock}

and this replaces:
    \begin{ucpythonblock}{}
for item in list:
  if conditional:
    expression
    \end{ucpythonblock}


\end{frame}

%-----------------------------------------------------------------------------------------

\begin{frame}[fragile]
  \frametitle{Useful list comprehension}

    \begin{ucpythonblock}{Filter one list into another (the ``old'' way)}
>>> v = [ x**2 for x in range(10) if x % 3 == 0 ]
>>> v
[0, 9, 36, 81]
    \end{ucpythonblock}

    \begin{ucpythonblock}{List comprehension (the ``pythonic'' way)}
newlist = [function(i) for i in oldlist if filter(i)]
    \end{ucpythonblock}

where \texttt{filter} and \texttt{function} just perform ``some'' operation on the list elements. Basically, the syntax is:
    \begin{ucpythonblock}{}
[ expression for item in list if conditional ] 
    \end{ucpythonblock}

and this replaces:
    \begin{ucpythonblock}{}
for item in list:
  if conditional:
    expression
    \end{ucpythonblock}


\end{frame}

%-----------------------------------------------------------------------------------------

\begin{frame}[fragile]
  \frametitle{Hello Gaussian!}

\begin{ucpythonblock}{Basic but useful code example}
import numpy as np
import matplotlib.pyplot as plt

def p(x):
    return np.exp(-x**2)
    
#let's plot it
x = np.linspace(-3,3,100)
y = p(x)
plt.plot(x,y)
plt.show()
\end{ucpythonblock}
  
  \pause

  \begin{textblock}{12}(8,-7)
    \includegraphics[width=0.55\textwidth]{Gaussian.png}
  \end{textblock}  
  
  \pause
  
  But what about that \texttt{linspace} thingy? Google it! (\href{https://docs.scipy.org/doc/numpy-1.15.0/reference/generated/numpy.linspace.html}{numpy docs})
  
  \texttt{numpy.linspace(start, stop, num=50, endpoint=True, retstep=False, dtype=None)}
  
  ``Returns num evenly spaced samples, calculated over the interval [start, stop].''
  
\end{frame}

%-----------------------------------------------------------------------------------------
\subsection[Jupyter notebooks]{Jupyter notebooks}
%-----------------------------------------------------------------------------------------

\begin{frame}[fragile]
  \frametitle{Jupyter notebooks}
  \framesubtitle{Interactive, web-based, integrated code and documentation environment}

  We will be following-up with more technical practice with python, but I want to introduce you to the resources that we'll be using this quarter for many of our examples and projects: \href{http://jupyter.org/}{Jupyter notebooks}.

  \centering
  \includegraphics[width=0.85\textwidth]{JupyterNotebooks.png}
  
\end{frame}

%-----------------------------------------------------------------------------------------

\begin{frame}[fragile]
  \frametitle{Jupyter notebooks on the PSD Binder platform}

  Built for Jupyter notebooks! \url{https://binderhub.pile.uchicago.edu/}

  \begin{figure}
  \centering
  \includegraphics<1>[width=0.75\textwidth]{Binder-Login.png}
  \includegraphics<2>[width=0.75\textwidth]{Binder-Startup.png}
  \includegraphics<3>[width=0.75\textwidth]{Binder-Jupyter.png}
  \includegraphics<4>[width=0.75\textwidth]{Binder-Jupyter-Files.png}
  \includegraphics<5>[width=0.75\textwidth]{Binder-Jupyter-Notebook.png}
  \end{figure}

\end{frame}

%==========================================================================================
\section[External Resources]{External Resources}
%==========================================================================================

%-----------------------------------------------------------------------------------------
\subsection[Textbooks]{Textbooks}
%-----------------------------------------------------------------------------------------

\begin{frame}[shrink=15]
  \frametitle{Textbooks that I suggest}

  \begin{table}[htbp]
  \centering
  \begin{tabular}{b{0.10\textwidth} b{3cm} b{5cm} b{3.5cm} }
    \hline
     & Author & Textbook & Comments \\ 
    \hline
    \includegraphics[width=0.10\textwidth]{Textbook-KN.png} & Kinder \& Nelson (\href{http://physicalmodelingwithpython.blogspot.com/}{link}) & A Student's Guide to Python for Physical Modeling & Very good for python examples, not great for physics principles \\ 
    \includegraphics[width=0.10\textwidth]{Textbook-Franklin.jpg} & Franklin (\href{http://www.reed.edu/physics/courses/P367.F14oldX!/downloads/index.html}{link}) & Computational Methods for Physics & Amazing for physical principles, no python \\ 
    \includegraphics[width=0.10\textwidth]{Textbook-LPB.jpg} & Landau, Paez, Bordeianu (\href{http://physics.oregonstate.edu/~landaur/Books/CPbook/Codes/}{link})  & Computational Physics, Problem Solving with Python & Great for both physics and python \\ 
    \includegraphics[width=0.10\textwidth]{Textbook-Halterman.pdf} & Halterman (\href{http://python.cs.southern.edu/cppbook/progcpp.pdf}{link}) & Fundamentals of \texttt{C++} Programming & Great for physical principles, no python \\ 
    \hline
  \end{tabular}
  \end{table}
  
  
\end{frame}

%-----------------------------------------------------------------------------------------

\begin{frame}%[shrink=15]
  \frametitle{Computer Science Instructional Lab (CSIL)}
  \framesubtitle{\url{https://csil.cs.uchicago.edu/index.html}}

  \begin{itemize}
    \item \textbf{Location:} first floor of Crerar (back right by stairs)
    \item \textbf{PHYS 250 Lab Hours:}
    \begin{itemize}
      \item Tue 7:00-9:00 pm in CSIL 1 \& 2 (Apple OSX)
      \item Wed 2:30-3:50 pm in CSIL 1 \& 2 (Apple OSX)
      \item Wed 7:00-9:00 pm in CSIL 1 \& 2 (Apple OSX)
    \end{itemize}
    \item \textbf{CSIL Minicourses}    
    \begin{itemize}
      \item Linux and \git minicourses available through CSIL
      \item additional tutorials are possible
      \item \url{https://csil.cs.uchicago.edu/minicourses}
    \end{itemize}
  \end{itemize}
    
  \centering \includegraphics[width=0.50\textwidth]{CSIL.png}
    
\end{frame}

%-----------------------------------------------------------------------------------------

\begin{frame}%[shrink=15]
  \frametitle{Research Computing Center (RCC)}
  \framesubtitle{\url{https://rcc.uchicago.edu/}}

  In addition to the short ``minicourse'' from CSIL, There is the possibility of holding longer tutorials (2-3 hrs) conducted by the RCC or CSIL for the following topics:

  \begin{itemize}
    \item Introduction to Linux and the RCC
    \item Introduction to Python
  \end{itemize}
    
  These are much more in-depth and would be a jumpstart to the quarter if you are interested. Please make sure to fill out the survey if you are interested:
  
  \begin{itemize}
    \item \url{https://forms.gle/J43bBZk5JCAoZLbH9}
  \end{itemize}
    
\end{frame}




%%==========================================================================================
%\section[Conclusions]{Conclusions}%==========================================================================================
%
%\begin{frame}%[shrink=1]
%  \frametitle{Conclusions}
%
%  \begin{itemize}
%    \item Something
%  \end{itemize}
%  
%\end{frame}

%==========================================================================================
%==========================================================================================
\end{document}
