%----------------------------------------------------
% Packages to use
%----------------------------------------------------
\documentclass[12pt]{article}
\usepackage{hyperref}
\usepackage[pdftex]{graphicx}
\usepackage{multirow}
\usepackage{xspace}

\hypersetup{
  colorlinks=true,
  linkcolor=red,
  citecolor=red,
  urlcolor=blue
}

%----------------------------------------------------
% Page Setup
%----------------------------------------------------
\usepackage{setspace}
\usepackage[margin=0.90in]{geometry}
\linespread{1.3}
%\doublespacing
%\addtolength{\hoffset}{-2.5cm}
%\addtolength{\voffset}{-2.5cm}
%\addtolength{\textwidth}{2.0cm}
%\addtolength{\textheight}{4.0cm}

%----------------------------------------------------
% New commands
%----------------------------------------------------
\def\undertilde#1{\mathord{\vtop{\ialign{##\crcr
$\hfil\displaystyle{#1}\hfil$\crcr\noalign{\kern1.5pt\nointerlineskip}
$\hfil\tilde{}\hfil$\crcr\noalign{\kern1.5pt}}}}}

\newcommand{\bluetext}[1] {\textcolor{blue}{#1}}

\newcommand{\angstrom}{\mbox{\normalfont\AA}\xspace}
\newcommand{\Qenc}    {\ensuremath{Q_{\rm enc}}\xspace}
\newcommand{\Qfenc}   {\ensuremath{Q_{f,\rm enc}}\xspace}
\newcommand{\Ienc}    {\ensuremath{I_{\rm enc}}\xspace}
\newcommand{\Ifenc}   {\ensuremath{I_{f, \rm enc}}\xspace}
\newcommand{\epn}     {\ensuremath{\varepsilon_{0}}\xspace}
\newcommand{\epr}     {\ensuremath{\varepsilon_{r}}\xspace}
\newcommand{\ep}      {\ensuremath{\varepsilon}\xspace}
\newcommand{\muno}    {\ensuremath{\mu_{0}}\xspace}
\newcommand{\mun}     {\ensuremath{\mu_{0}}\xspace}
\newcommand{\mur}     {\ensuremath{\mu_{r}}\xspace}
\newcommand{\mucoeff} {\ensuremath{\frac{\mun}{4\pi}}\xspace}
\newcommand{\Pol}     {\ensuremath{\vec{P}}\xspace}
\newcommand{\Surf}    {\ensuremath{\mathcal{S}}\xspace}
\newcommand{\rhob}    {\ensuremath{\rho_{b}}\xspace}
\newcommand{\rhof}    {\ensuremath{\rho_{f}}\xspace}
\newcommand{\sigb}    {\ensuremath{\sigma_{b}}\xspace}
\newcommand{\sigch}   {\ensuremath{\sigma_{\rm charged}}\xspace}
\newcommand{\sigfr}   {\ensuremath{\sigma_{\rm free}}\xspace}
\newcommand{\fvec}    {\ensuremath{\vec{f}}\xspace}
\newcommand{\Fvec}    {\ensuremath{\vec{F}}\xspace}
\newcommand{\Evec}    {\ensuremath{\vec{E}}\xspace}
\newcommand{\Ezero}   {\ensuremath{E_{0}}\xspace}
\newcommand{\EvecC}   {\ensuremath{\tilde{\vec{E}}}\xspace}
\newcommand{\Etil}    {\ensuremath{\tilde{E}}\xspace}
\newcommand{\EzeroC}  {\ensuremath{\tilde{\Ezero}}\xspace}
\newcommand{\Dvec}    {\ensuremath{\vec{D}}\xspace}
\newcommand{\Bvec}    {\ensuremath{\vec{B}}\xspace}
\newcommand{\Bzero}   {\ensuremath{B_{0}}\xspace}
\newcommand{\BvecC}   {\ensuremath{\tilde{\vec{B}}}\xspace}
\newcommand{\Btil}    {\ensuremath{\tilde{B}}\xspace}
\newcommand{\BzeroC}  {\ensuremath{\tilde{\Bzero}}\xspace}
\newcommand{\Avec}    {\ensuremath{\vec{A}}\xspace}
\newcommand{\Cvec}    {\ensuremath{\vec{C}}\xspace}
\newcommand{\Hvec}    {\ensuremath{\vec{H}}\xspace}
\newcommand{\Vvec}    {\ensuremath{\vec{v}}\xspace}
\newcommand{\vvec}    {\ensuremath{\vec{v}}\xspace}
\newcommand{\Ivec}    {\ensuremath{\vec{I}}\xspace}
\newcommand{\Jvec}    {\ensuremath{\vec{J}}\xspace}
\newcommand{\Jvecf}   {\ensuremath{\vec{J}_f}\xspace}
\newcommand{\Kvec}    {\ensuremath{\vec{K}}\xspace}
\newcommand{\Kvecf}   {\ensuremath{\vec{K}_f}\xspace}
\newcommand{\mvec}    {\ensuremath{\vec{m}}\xspace}
\newcommand{\Mvec}    {\ensuremath{\vec{M}}\xspace}
\newcommand{\Svec}    {\ensuremath{\vec{S}}\xspace}
\newcommand{\Del}     {\ensuremath{\nabla}\xspace}
\newcommand{\Delvec}  {\ensuremath{\vec{\Del}}\xspace}
\newcommand{\dtau}    {\ensuremath{d\tau}\xspace}
\newcommand{\davec}   {\ensuremath{d\vec{a}}\xspace}
\newcommand{\dlvec}   {\ensuremath{d\vec{\ell}}\xspace}
\newcommand{\Rvec}    {\ensuremath{\vec{R}}\xspace}
\newcommand{\rvec}    {\ensuremath{\vec{r}}\xspace}
\newcommand{\rvecp}   {\ensuremath{\rvec^{\,\prime}}\xspace}
\newcommand{\nhat}    {\ensuremath{\hat{n}}\xspace}
\newcommand{\xhat}    {\ensuremath{\hat{x}}\xspace}
\newcommand{\yhat}    {\ensuremath{\hat{y}}\xspace}
\newcommand{\zhat}    {\ensuremath{\hat{z}}\xspace}
\newcommand{\rhat}    {\ensuremath{\hat{r}}\xspace}
\newcommand{\shat}    {\ensuremath{\hat{s}}\xspace}
\newcommand{\khat}    {\ensuremath{\hat{k}}\xspace}
\newcommand{\kvec}    {\ensuremath{\vec{k}}\xspace}
\newcommand{\kC}      {\ensuremath{\tilde{k}}\xspace}
\newcommand{\kvecC}   {\ensuremath{\tilde{\kvec}}\xspace}
\newcommand{\phihat}  {\ensuremath{\hat{\phi}}\xspace}
\newcommand{\Phihat}  {\ensuremath{\hat{\Phi}}\xspace}
\newcommand{\thetahat}{\ensuremath{\hat{\theta}}\xspace}
\newcommand{\Rhat}    {\ensuremath{\hat{R}}\xspace}
\newcommand{\chim}    {\ensuremath{\chi_m}\xspace}
\newcommand{\chie}    {\ensuremath{\chi_e}\xspace}
\newcommand{\veloc}   {\ensuremath{\vec{v}}\xspace}
\newcommand{\avec}    {\ensuremath{\vec{a}}\xspace}
\newcommand{\ahat}    {\ensuremath{\hat{a}}\xspace}
\newcommand{\EMF}     {\ensuremath{\mathcal{E}_{\rm EMF}}\xspace}
\newcommand{\Flux}    {\ensuremath{\Phi_{B}}\xspace}
\newcommand{\umech}   {\ensuremath{u_{\rm mech}}\xspace}
\newcommand{\uem}     {\ensuremath{u_{\rm em}}\xspace}
\newcommand{\uch}     {\ensuremath{u_{\rm charges}}\xspace}
\newcommand{\ufields} {\ensuremath{u_{\rm fields}}\xspace}
\newcommand{\STensor} {\ensuremath{\undertilde{\mathbf{T}}}\xspace}
\newcommand{\dskin}   {\ensuremath{d_{\mathrm{skin}}\xspace}}
\newcommand{\thetac}  {\ensuremath{\theta_{C}}\xspace}
\newcommand{\pvec}    {\ensuremath{\vec{p}}\xspace}
\newcommand{\upp}     {\ensuremath{^{\prime}}\xspace}
\newcommand{\timer}   {\ensuremath{t_r}\xspace}
\newcommand{\timeR}   {\ensuremath{t_R}\xspace}
\newcommand{\betavec} {\ensuremath{\vec{\beta}}\xspace}
\newcommand{\betahat} {\ensuremath{\hat{\beta}}\xspace}

\newcommand{\thetacrit}{\ensuremath{\theta_{\mathrm{crit}}}\xspace}

\newcommand{\kzmwt}   {\ensuremath{kz - \omega t}\xspace}
\newcommand{\kzpwt}   {\ensuremath{kz + \omega t}\xspace}
\newcommand{\kxmwt}   {\ensuremath{kx - \omega t}\xspace}
\newcommand{\kxpwt}   {\ensuremath{kx + \omega t}\xspace}

\newcommand{\xbar}              {\ensuremath{\overline{x}}\xspace}
\newcommand{\ybar}              {\ensuremath{\overline{y}}\xspace}
\newcommand{\zbar}              {\ensuremath{\overline{z}}\xspace}
\newcommand{\tbar}              {\ensuremath{\overline{t}}\xspace}
\newcommand{\rbar}              {\ensuremath{\overline{r}}\xspace}
\newcommand{\Vbar}              {\ensuremath{\overline{V}}\xspace}
\newcommand{\Abar}              {\ensuremath{\overline{A}}\xspace}
\newcommand{\phibar}            {\ensuremath{\overline{\phi}}\xspace}
\newcommand{\rframe}            {\ensuremath{\mathcal{S}}\xspace}
\newcommand{\vframe}            {\ensuremath{\mathcal{\overline{S}}}\xspace}  

\newcommand{\Jmu}               {\ensuremath{J^{\mu}}\xspace}
\newcommand{\Jnu}               {\ensuremath{J^{\nu}}\xspace}
\newcommand{\Fmunu}             {\ensuremath{F^{\mu\nu}}\xspace}
\newcommand{\Gmunu}             {\ensuremath{G^{\mu\nu}}\xspace}
\newcommand{\Fnumu}             {\ensuremath{F^{\nu\mu}}\xspace}
\newcommand{\Gnumu}             {\ensuremath{G^{\nu\mu}}\xspace}

\newcommand{\degs}              {\ensuremath{^{\circ}}\xspace}

\newcommand{\Ecoeff}            {\ensuremath{\frac{1}{4\pi\epn}}\xspace}
\newcommand{\Bcoeff}            {\ensuremath{\frac{\mun}{4\pi}}\xspace}

\newcommand{\avg}[1]            {\ensuremath{\langle #1 \rangle}\xspace}

\newcommand{\Curl}[1]           {\ensuremath{\Delvec \times #1}\xspace}
\newcommand{\Div}[1]            {\ensuremath{\Delvec \cdot #1}\xspace}

\newcommand{\partialx}[1]       {\ensuremath{\frac{\partial #1}{\partial x}}\xspace}
\newcommand{\partialy}[1]       {\ensuremath{\frac{\partial #1}{\partial y}}\xspace}
\newcommand{\partialz}[1]       {\ensuremath{\frac{\partial #1}{\partial z}}\xspace}
\newcommand{\partialt}[1]       {\ensuremath{\frac{\partial #1}{\partial t}}\xspace}

\newcommand{\partialnu}         {\ensuremath{\partial_{\nu}}\xspace}
\newcommand{\partialmu}         {\ensuremath{\partial_{\mu}}\xspace}
\newcommand{\partialxnu}[1]     {\ensuremath{\frac{\partial #1}{\partial x^{\nu}}}\xspace}
\newcommand{\partialxmu}[1]     {\ensuremath{\frac{\partial #1}{\partial x^{\mu}}}\xspace}
\newcommand{\partialxzero}[1]   {\ensuremath{\frac{\partial #1}{\partial x^{0}}}\xspace}
\newcommand{\partialxone}[1]    {\ensuremath{\frac{\partial #1}{\partial x^{1}}}\xspace}
\newcommand{\partialxtwo}[1]    {\ensuremath{\frac{\partial #1}{\partial x^{2}}}\xspace}
\newcommand{\partialxthr}[1]    {\ensuremath{\frac{\partial #1}{\partial x^{3}}}\xspace}

\newcommand{\partialr}[1]       {\ensuremath{\frac{\partial #1}{\partial r}}\xspace}
\newcommand{\partialtheta}[1]   {\ensuremath{\frac{\partial #1}{\partial \theta}}\xspace}
\newcommand{\partialphi}[1]     {\ensuremath{\frac{\partial #1}{\partial \phi}}\xspace}

\newcommand{\partialxsq}[1]     {\ensuremath{\frac{\partial^2 #1}{\partial x^2}}\xspace}
\newcommand{\partialysq}[1]     {\ensuremath{\frac{\partial^2 #1}{\partial y^2}}\xspace}
\newcommand{\partialzsq}[1]     {\ensuremath{\frac{\partial^2 #1}{\partial z^2}}\xspace}
\newcommand{\partialtsq}[1]     {\ensuremath{\frac{\partial^2 #1}{\partial t^2}}\xspace}

%----------------------------------------------------
% Header and Footer Information
%----------------------------------------------------
\usepackage{fancyhdr}
\pagestyle{fancy}
\lhead{PHYS 227, Spring Quarter 2018}
\rhead{\today, v1.0}

%----------------------------------------------------
% Course Number, Course Title, Prof Name, Location
%----------------------------------------------------
\title{\sc Physics 227: Intermediate Electricity and Magnetism II\\Learning Goals}
\author{\textbf{Instructor:} David W. Miller, PRC 245, \\ 773-702-7671, \href{mailto:David.W.Miller@uchicago.edu}{David.W.Miller@uchicago.edu}}
\date{}


\begin{document}

\maketitle

\thispagestyle{fancy}

%----------------------------------------------------
% Learning Goals
%----------------------------------------------------

\noindent This document differs from a syllabus in that it is not a list of nouns (topics covered), but a list of verbs that indicate what students are \textit{expected to be able \textbf{to do} after taking this course}.

%--------------------------------
\subsection*{Electrodynamics}
%--------------------------------

\begin{itemize}
  \item Determine the EMF, \EMF, from a changing magnetic field inside of a conducting loop
  \item Be able to apply Lenz's law to determine the orientation of the induced current, \Ivec, and resulting magnetic flux, \Flux, in the case of magnetic induction
\end{itemize}

%--------------------------------
\subsection*{Maxwell's Equations}
%--------------------------------

\begin{itemize}
  \item Recognize the addition of the displacement current to Ampere's Law
  \item Write down each of Maxwell's equations in both a vacuum and a linear medium with and without free charge and current densities and identify the key features related to time dependence and sources ($\rho$ and \Jvec)
\end{itemize}

%--------------------------------
\subsection*{Conservation Laws and Energy}
%--------------------------------

\begin{itemize}
  \item Employ the conservation of energy, momentum, and charge in the context of time varying fields
  \item Calculate the rate of change of energy in an arbitrary EM field, including it's direction, via Poynting's Vector for both purely real fields as well as field with a complex phase
  \item Compute the forces acting on a surface due to EM fields using Maxwell's stress tensor
  \item Recognize the relationship between Ohm's law and the capacity of EM fields to carry energy and momentum
  \item Understand the connection between classical EM waves and the quantum description of photons and their spin in the context of angular momentum
\end{itemize}

%--------------------------------
\subsection*{Electromagnetic Wave Equation and Functions}
%--------------------------------

\begin{itemize}
  \item Write down the wave equation in both 1D and 3D and evaluate whether an arbitrary field wave function satisfies the wave equation
  \item Determine the velocity of a wave given the coefficients of the wave equation
  \item Establish the vectorial relationship between an arbitrary electric field wave function and its magnetic field counterpart using the known relationship between E and B for an EM wave
  \item Determine the radiation pressure associated with an EM wave for given E and B field wave functions
  \item Specify the polarization direction when given the vectorial form of the E and B field wave functions
\end{itemize}

%--------------------------------
\subsection*{Electromagnetic Waves in Vacuum and Materials}
%--------------------------------

\begin{itemize}  
  \item Describe the notion of effective velocity of light in a material with a real index of refraction greater than 1.0
  \item Match boundary conditions at an interface between two non-conducting materials with different indices of refraction
  \item Understand and use Fresnel's equations for a wave impinging on a boundary at normal incidence and determine the reflected and transmitted amplitudes relative to the incident amplitude
  \item Understand and use Fresnel's equation for a wave impinging on a boundary at oblique incidence  and determine the reflected, transmitted amplitudes relative to the incident amplitude, in the case of polarization both parallel and perpendicular to the plane of incidence
  \item Explain the gross features that differentiate the cases of oblique incidence reflection and transmission for polarization both parallel and perpendicular to the place of incidence
  \item Use and explain Brewster's angle for the case of polarization parallel to the place of incidence
  \item Describe the dissipation time of free charge in the case that an EM wave impinges on a conducting surface
  \item Describe the depth at which the dissipation of free charge occurs in terms of the skin depth in the case case that an EM wave impinges on a conducting surface
  \item Use both the real and imaginary components of complex indices of refraction and wavenumbers in the context of an EM wave impinging on a conducting surface
  \item Derive the critical angle at which total internal reflection takes place from Snell's law and describe the impact of this angle on optical and EM devices
  \item Discuss the phenomenon of evanescent waves and the features of the attenuation experienced by an EM wave in a material in which such a wave propagates
\end{itemize}

%--------------------------------
\subsection*{Guided Electromagnetic Waves}
%--------------------------------

\begin{itemize}  
  \item Understand the notion of a ``guide'' wavelength or ``group'' wavelength ($\lambda_g$), how this translates into a group velocity ($v_g$), and how to use the relationship between the group wavenumber ($k_g$), nominal wavenumber ($k_0$), and cutoff wavenumber ($k_c$) or wavelength ($\lambda_c$)
  \item Determine the conditions placed on the allowed values of the wavenumber in all three dimensions in the case of EM waves guided by conducting surfaces
  \item Apply the separation of variables technique for solving the wave equation in a wave guide to find an appropriate wave function for a wave that satisfies the conditions imposed on it by a rectangular wave guide 
  \item Describe the quantitative and qualitative differences between a TE, TM and TEM mode wave in a wave guide
  \item Be able to calculate the Poynting vector for a TE, TM, and TEM wave in a waveguide
  \item Describe the key qualitative and quantitative differences between a wave guide that is able to support a TEM mode wave, and one that is not able to support a TEM mode wave
\end{itemize}

%--------------------------------
\subsection*{Potential formulation of Maxwell's Equations}
%--------------------------------

\begin{itemize}  
  \item Be able to write the \Evec and \Bvec fields in terms of the scalar and vector potentials, $\phi$ and \Avec
  \item Be able to calculate the \Evec and \Bvec fields from a given set of scalar and vector potentials, $\phi$ and \Avec
  \item Derive a wave equation in terms of $\phi$ and \Avec from Maxwell's equations written in terms of the potentials
  \item Apply the Lorentz gauge and obtain the canonical forms of the inhomogenous Maxwell's equations in terms of the scalar and vector potentials, $\phi$ and \Avec, and the source terms, $\rho$ and \Jvec
  \item Explain the relationship and differences between the Lorentz gauge and the Coulomb gauge
\end{itemize}

%--------------------------------
\subsection*{Fields and energy transport due to moving and accelerated charged particles}
%--------------------------------

\begin{itemize}  
  \item Write down the definition of the retarded time and explain why it is necessary in the context of radiation from charge and current sources
  \item Calculate $\phi$ and \Avec, and then the resulting \Evec and \Bvec, for basic canonical systems (e.g. wire) using the principles of retarded time and the potential formulations of Maxwell's equations
  \item Perform the requisite change of variables in order to translate the scalar and vector potentials into the Li\'enard--Wiechert potentials for a point charge moving with a given velocity 
  \item Be able to write down the functional dependence of \Evec and \Bvec on \Rvec (the distance between a particle at a given retarded time and a field point)
  \item Identify and manipulate the relevant terms in the Li\'enard--Wiechert fields, \Evec and \Bvec, at various limits (e.g. non-relativistic, small \avec, etc)
  \item Describe the relationship between \Evec, \Bvec for an accelerated charge
  \item Compute the Poynting vector for an accelerated charged particle and identify the quantitative relationship between the resulting magnitude of the energy flux per unit time and \Rvec
  \item Understand the concept of radiation in terms of energy transport to $\infty$
  \item Be able to draw the \Evec field for a particle moving with constant velocity as $|\vec{v}|\rightarrow c$
  \item Compute the total power emitted by an accelerated charged particle with a given $\Evec_{a}$ (i.e. the term uniquely due  to acceleration in the \Evec for a charged particle moving with a velocity and acceleration). 
\end{itemize}

%--------------------------------
\subsection*{Multipole radiation and antennas}
%--------------------------------

\begin{itemize}   
  \item Write down (quantitatively) the three approximations typically made as part of the determination of the radiation due to a dipole: dipole approximation, local non-relativistic approximation, radiation-zone approximation.
  \item Determine the scalar and vector potentials for a simple dipole without any approximations
  \item Be able to apply the three approximations mentioned above in the calculation of of the scalar and vector potentials for a dipole
  \item Determine the \Evec and \Bvec fields for a dipole given the scalar and vector potentials
  \item Derive the expression for radiation from a dipole given the \Evec and \Bvec fields
  \item Express the functional dependence of the energy flux per unity time for a dipole, as well as the total power, on the angular frequency, $\omega$, the dipole moment, $p_0$ and the angle between the dipole and the field point, $\theta$
  \item Draw the dipole radiation pattern described by the time-averaged Poynting vector. 
  \item Determine the half-wave antenna 
\end{itemize}


%--------------------------------
\subsection*{Relativistic Electrodynamics}
%--------------------------------

\begin{itemize}   
  \item Be able to state the general guiding principle behind the theory of special relativity
  \item Transform the time and spatial coordinates in a moving inertial reference frame to one that is at rest with respect to the observer.
  \item Be able to understand and use Einstein notation.
  \item Compute a dot product in 4-vector notation
  \item Compute derivatives in 4-vector notation
  \item Derive the inhomogeneous Maxwell's equations from the 4-vector definitions of the scalar and vector potentials.
\end{itemize}

\end{document}