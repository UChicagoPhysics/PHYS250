%----------------------------------------------------
% Setup Beamer
%----------------------------------------------------
\documentclass[hyperref={colorlinks=true}]{beamer}

%----------------------------------------------------
% Packages to use
%----------------------------------------------------
\input{../packages.sty}

%\usetikzlibrary{external}
%\tikzexternalize % activate!

%----------------------------------------------------
% Setup Theme
%----------------------------------------------------
\input{../theme.sty}

%----------------------------------------------------
% Table of Contents at each section transition
%----------------------------------------------------

\AtBeginSection[]
{
   \begin{frame}
       \frametitle{Outline}
       \setcounter{tocdepth}{2}
       \tableofcontents[currentsection]
   \end{frame}
}

%----------------------------------------------------
% Colors
%----------------------------------------------------
\input{../mycolors.sty}

%----------------------------------------------------
% Style, formatting, and new commands
%----------------------------------------------------
\newcommand{\CourseYear}   {2025}
\newcommand{\CanvasURL}    {https://canvas.uchicago.edu/courses/66016}
\newcommand{\CanvasLink}   {\href{\CanvasURL}{\CanvasURL}}
\newcommand{\GitHubURL}    {https://github.com/UChicagoPhysics/PHYS250}
\newcommand{\GitHubLink}   {\href{\GitHubURL}{\GitHubURL}}
\newcommand{\PlatformURL}  {https://binderhub.pile.uchicago.edu/}
\newcommand{\PlatformLink} {\href{\PlatformURL}{\PlatformURL}}
\newcommand{\PiazzaURL}    {https://canvas.uchicago.edu/courses/66016/discussion\_topics}
\newcommand{\PiazzaLink}   {\href{\PiazzaURL}{\PiazzaURL}}
\input{../newcommands.sty}
\input{../EandMcommands.sty}

%----------------------------------------------------
% Set paths for plots and images
%----------------------------------------------------
\input{../paths.sty}

%----------------------------------------------------
% SETTINGS FOR THIS LECTURE
%----------------------------------------------------
\newcommand{\lecnum }  {Lecture 16}
\newcommand{\lecdate}  {November 29, 2024}
\newcommand{\topic}    {Neural Networks -- Part II}

%-----------------------------------------------------------------------------------------
% Title: [Column]{Title}
%-----------------------------------------------------------------------------------------
\title[PHYS 250 (Autumn 2025) -- \lecnum]{\topic}

%-----------------------------------------------------------------------------------------
% SubTitle: [Column]{Subtitle}
%-----------------------------------------------------------------------------------------
\subtitle{PHYS 250 (Autumn 2025) -- \lecnum}

%-----------------------------------------------------------------------------------------
% Author: [SubAuthor]{Author}
%-----------------------------------------------------------------------------------------
\author[D.W.~Miller]{David Miller}

%----------------------------------------------------
% Institute: [SubInst]{Institute}
%----------------------------------------------------
\institute[EFI, Chicago] 
{
  Department of Physics and the Enrico Fermi Institute\\
  University of Chicago
}

%----------------------------------------------------
% Institute: [SubInst]{Institute}
%----------------------------------------------------
\date[\lecdate]{\lecdate}

\subject{PHYS 250 Lecture}

\begin{document}

%==========================================================================================
% TITLE PAGE
%==========================================================================================

{
\begin{frame}
  \titlepage
\end{frame}
}

%==========================================================================================
\section[Reminders]{Reminders}
%==========================================================================================

%-----------------------------------------------------------------------------------------
\subsection[Reminders from Lecture 15]{Reminders from Lecture 15}
%-----------------------------------------------------------------------------------------

\begin{frame}%[shrink=10]
  \frametitle{Reminders from last time}

  We embarked on a whirlwind introduction to neural networks.
  
  \vspace{0.3cm}
  
  \begin{ucblock}{Neural networks and machine learning}
    \begin{itemize}
      \item \bluebf{Context and perspective} 
      \begin{itemize}
        \item We discussed the general issue of training computers to \textbf{discover, identify, and analyze patterns} of interest in datasets
        \item Categorized tasks that make use of this idea: \textbf{classification, regression, generation, clustering, anomaly detection}
      \end{itemize}
      \item \bluebf{Neural networks as a tool} 
      \begin{itemize}
        \item Introduced both the \textbf{modeling} perspective as well as the \textbf{biological} perspective on what a neural network achieves 
        \item Described the \textbf{structure and function} of a neuron
        \item Began discussing the \textbf{mathematical properties} of a neural network
      \end{itemize}
    \end{itemize}
  \end{ucblock}
  
  \mysp
  
  Today we will build our own networks! But first, I just wanted to follow-up on some points and questions from last time.

\end{frame}

%==========================================================================================
\section[Historical perspective]{Historical perspective}
%==========================================================================================

%-----------------------------------------------------------------------------------------
\subsection[Brief History of Machine Learning Generally]{Brief History of Machine Learning Generally}
%-----------------------------------------------------------------------------------------

\begin{frame}%[shrink=10]
  \frametitle{Brief history of machine learning}

  Taken from \href{http://innovationobservatory.com/node/243}{Harry Ide on InnovationLaboratory.com (18 May 2018)}:
  
  \begin{figure}
    \centering
    \includegraphics[width=\textwidth]{MLBreakthroughs-Timeline-to-2017.png}
  \end{figure}
  
  \only<2>{
    \begin{textblock}{12}(1.5,-8)
    \begin{ucblock}{Quote from Harry Ide on ``AI''}
      \textit{``The cutting edge might not be as sharp as many believe. AI has a nearly 70-year long history with some surprising backtracks and re-emergences among it.''}
    \end{ucblock}
    \end{textblock}
  }
\end{frame}


%-----------------------------------------------------------------------------------------
\subsection[Brief History of Neural Networks]{Brief History of Neural Networks}
%-----------------------------------------------------------------------------------------

\begin{frame}%[shrink=10]
  \frametitle{Brief history of neural networks}

  Taken from \href{https://www.slideshare.net/deview/251-implementing-deep-learning-using-cu-dnn/4}{this talk on SlideShare}:
  
  \begin{figure}
    \centering
    \includegraphics[width=\textwidth]{NN_Timeline.jpg}
  \end{figure}

\end{frame}


%==========================================================================================
\section[Structure of Neural Networks]{Structure of Neural Networks}
%==========================================================================================

%-----------------------------------------------------------------------------------------
\subsection[Single layer perceptron]{Single layer perceptron}
%-----------------------------------------------------------------------------------------

\begin{frame}%[shrink=10]
  \frametitle{Single layer perceptron}

  \begin{figure}
    \centering 
    \resizebox {0.7\columnwidth} {!} {
      \input{singlelayer_perceptron.tex}
    }
  \end{figure}
  
  \begin{itemize}
    \item \bluebf{$\vec{x} = (x_1, x_2, \cdots, x_n)$ is an input feature vector of length $n$}
    \begin{itemize}
      i.e. the attributes of the data, e.g. voltages
    \end{itemize}
    \item \bluebf{$\vec{w} = (w_1, w_2, \cdots, w_n)$ is the weight vector with $w_0$ reserved as a \textbf{bias}}
    \begin{itemize}
      \item becomes a matrix for multiple layers
    \end{itemize}
    \item \bluebf{$\Sigma$ indicates summation (or matrix mult.): $z = \sum w_i x_i$ ($x_0 = 1$)}
    \item \bluebf{$f$ is the activation function, or non-linearity: $f(z)$}
    \item \alertbf{$y = f(z)$ is the output}
  \end{itemize}
  
\end{frame}

%-----------------------------------------------------------------------------------------

\begin{frame}%[shrink=10]
  \frametitle{Sigmoid as activation function}

  As we discussed, a typical function for a \alertbf{single layer perceptron} is the \textbf{sigmoid}. 

  \begin{figure}
    \centering 
    \resizebox {0.8\columnwidth} {!} {
      \input{sigmoid-derivative.tex}
    }
  \end{figure}
  
  Here, we plot both the function itself, as well as its derivative, since that will be important when evaluating the \bluebf{backpropagation} of weights in order to update the neural network.

\end{frame}

%-----------------------------------------------------------------------------------------
\subsection[Training a single layer perceptron]{Training a single layer perceptron}
%-----------------------------------------------------------------------------------------


\begin{frame}%[shrink=10]
  \frametitle{Training a single layer perceptron}

  \begin{figure}
    \centering 
    \resizebox {0.7\columnwidth} {!} {
      \input{singlelayer_perceptron.tex}
    }
  \end{figure}
  
  Given $j$ objects $\vec{x}_j$ in dataset, each with \alertbf{known values of $f$}, $d_j$
  \begin{itemize}
    \item \bluebf{Calculate the output:} $y_j = f(\vec{w}\cdot\vec{x}_j)$
    \item \bluebf{Determine the error:} $\epsilon_j = d_j - y_j$
    \item \bluebf{Update the weights:} $w^{\mathrm{new}}_i = w_i + r ( \epsilon_j \cdot \vec{x}_j )_{i}$
  \end{itemize}
  
  Choosing the learning rate $r$ is where the derivative is used. It's not important for the single-layer perceptron, but is \alertbf{essential} for a network.
  
\end{frame}

%-----------------------------------------------------------------------------------------
\subsection[Training a Multi-Layer Perceptron (MLP)]{Training a Multi-Layer Perceptron (MLP)}
%-----------------------------------------------------------------------------------------

\begin{frame}[shrink=20]
  \frametitle{Multi-layer perceptron (MLP)}

  \vspace{-0.8cm}

  \begin{figure}
    \centering 
    \resizebox {\columnwidth} {!} {
      \input{multilayer_perceptron.tex}
    }
  \end{figure}
  
  Given $j$ objects $\vec{I}_j$ in dataset, each with features $\vec{I} = (I_1, I_2, \cdots, I_n)$ and \alertbf{known outputs $\vec{d}_j$ at each output node $o$}, $\vec{d} = (d_1, d_2, \cdots, d_o)$
  \begin{itemize}
    \item \bluebf{Calculate the $h$ outputs of hidden layer:} $v_h = f(\sum_i w_{ih} I_i)$
    \item \bluebf{Calculate the $o$ outputs of output layer:} $y_o = f(\sum_h w_{ho} v_h)$
    \item \bluebf{Determine the error at output each node $o$:} $\epsilon_o = d_o - y_o$
    \item \bluebf{Determine the total error for data object $j$:} $\mathcal{E}_j = \frac{1}{2} \sum_o \epsilon_o^{2}$
    \item \bluebf{Determine change in weights for output neuron $y_o$:} $\Delta w_{oh}=-\eta {\frac {\partial {\mathcal{E}}}{\partial z_{o}}}v_{h} = \eta \epsilon_o \, f^{\prime}(z_{o})$
  \end{itemize}
    
\end{frame}

%-----------------------------------------------------------------------------------------

\begin{frame}[shrink=20]
  \frametitle{LeCun, Bengio, Hinton, ``Deep learning''}
  \framesubtitle{\href{https://www.nature.com/articles/nature14539\#f1}{Nature volume 521, pages 436-444 (28 May 2015)} }

  \vspace{-0.3cm}

  \begin{figure}
    \centering 
    \includegraphics[width=\textwidth]{BackPropagation-LeCun-Nature.jpg}
  \end{figure}
  

    
\end{frame}

%-----------------------------------------------------------------------------------------

%\begin{frame}%[shrink=20]
%
%  \begin{figure}
%    \centering 
%    \resizebox {0.7\columnwidth} {!} {
%      \input{singlelayer_perceptron_3.tex}
%    }
%  \end{figure}
%
%\end{frame}

  
%==========================================================================================
%==========================================================================================
\end{document}
